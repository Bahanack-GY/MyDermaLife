\documentclass[11pt,a4paper]{article}
\usepackage[utf8]{inputenc}
\usepackage[T1]{fontenc}
\usepackage[margin=1in]{geometry}
\usepackage{titlesec}
\usepackage{hyperref}
\usepackage{xcolor}
\usepackage{enumitem}
\usepackage{fancyhdr}
\usepackage{booktabs}
\usepackage[french]{babel}

% Colors
\definecolor{primary}{RGB}{0, 51, 102}
\definecolor{accent}{RGB}{204, 85, 0}

% Page Setup
\pagestyle{fancy}
\fancyhf{}
\rhead{\today}
\lhead{MyDermaLife - Rapport de Développement}
\cfoot{\thepage}

% Title Format
\titleformat{\section}
{\normalfont\Large\bfseries\color{primary}}
{\thesection}{1em}{}

\titleformat{\subsection}
{\normalfont\large\bfseries\color{primary}}
{\thesubsection}{1em}{}

\title{
    \vspace{-1in}
    \huge \textbf{Rapport de Développement Quotidien} \\
    \large Projet : Plateforme MyDermaLife
}
\author{Équipe de Développement}
\date{2 Février 2026}

\begin{document}

\maketitle
\thispagestyle{empty}

\hrule
\vspace{1em}

\section*{Résumé Exécutif}
Ce rapport détaille les activités de développement menées le 2 février 2026. L'objectif principal était de standardiser le modèle commercial (tarification), d'améliorer l'expérience de navigation avec une section unifiée "Comment ça marche", et de déployer des pages essentielles pour la confiance des utilisateurs (Sécurité, Parcours Patient). De plus, une internationalisation complète (Anglais/Français) a été mise en œuvre pour tous les nouveaux modules.

\section{Réalisations Clés}

\subsection{1. Standardisation de la Tarification}
Afin d'assurer cohérence et transparence sur la plateforme, le modèle de tarification des consultations a été unifié.
\begin{itemize}
    \item \textbf{Prix Unique :} Toutes les consultations dermatologiques sont désormais fixées à \textbf{25 000 FCFA}.
    \item \textbf{Intégrité des Données :} La base de données des médecins (`doctors.ts`) a été mise à jour pour refléter ce changement, corrigeant spécifiquement le tarif du Dr. Jean-Paul Mbarga.
    \item \textbf{Optimisation UI :} Les filtres de prix ont été retirés de l'interface de \textit{Recherche de Médecins} pour éviter toute confusion, et la \textit{Page de Consultation} a été simplifiée pour afficher ce tarif unique.
\end{itemize}

\subsection{2. Améliorations de l'Architecture et de la Navigation}
La structure de navigation de l'application a été considérablement améliorée pour mieux guider les utilisateurs.
\begin{itemize}
    \item \textbf{Menu de Navigation :} La barre de navigation intègre désormais un menu déroulant complet \textbf{"Comment ça marche"}.
    \item \textbf{Découverte Médecins :} Un menu déroulant \textbf{"Dermatologues"} a été ajouté, offrant un accès direct aux profils de l'équipe médicale, aux certifications et aux spécialités traitées.
\end{itemize}

\subsection{3. Nouveaux Modules d'Information}
Quatre nouvelles pages stratégiques ont été développées et intégrées pour renforcer la confiance et clarifier le processus :
\begin{itemize}[label=$\cdot$]
    \item \textbf{Parcours Patient :} Un guide étape par étape détaillant la prise de rendez-vous, la consultation et le soin.
    \item \textbf{Infos Consultation en Ligne :} Un guide technique et opérationnel pour les patients (équipement requis, durée, livrables post-consultation).
    \item \textbf{Sécurité \& Confidentialité :} Une page de conformité dédiée mettant en avant l'hébergement HDS, le chiffrement de bout en bout et le secret médical.
    \item \textbf{Tarifs & Remboursement :} Un hub de transparence détaillant le tarif de 25 000 FCFA, les moyens de paiement acceptés (Mobile Money, Carte), et la génération de factures pour assurance.
\end{itemize}

\subsection{4. Internationalisation (I18n)}
Pour soutenir les objectifs bilingues de la plateforme, une localisation complète a été implémentée pour les nouvelles fonctionnalités.
\begin{itemize}
    \item \textbf{Architecture :} Toutes les nouvelles pages ont été refactorisées pour utiliser le hook `useTranslation`, supprimant le texte codé en dur.
    \item \textbf{Contenu :} Des clés de traduction complètes ont été ajoutées à `en.json` (Anglais) et `fr.json` (Français) couvrant tout le nouveau contenu "Comment ça marche".
\end{itemize}

\section{Améliorations Techniques}
\begin{itemize}
    \item \textbf{Hygiène du Code :} Suppression des imports inutilisés et du code mort dans plusieurs composants (`HomePage`, `DoctorSearchPage`, etc.) pour améliorer la maintenabilité.
    \item \textbf{Gestion d'État :} Correction de l'implémentation de l'état React dans la `Navbar` pour assurer des interactions fluides avec les menus déroulants.
    \item \textbf{Routage :} Toutes les nouvelles pages ont été correctement enregistrées dans le routeur principal de l'application (`App.tsx`).
\end{itemize}

\section{Conclusion}
Les modifications effectuées aujourd'hui élèvent significativement le professionnalisme de la plateforme MyDermaLife. La tarification unifiée élimine les frictions, tandis que les nouvelles pages éducatives apportent la transparence et les signaux de confiance nécessaires à une plateforme médicale. L'application est désormais pleinement bilingue sur ses flux d'information principaux.

\vfill
\textit{Généré par l'Assistant IA le \today}

\end{document}
